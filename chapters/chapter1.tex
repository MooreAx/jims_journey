\chapter{Part 1}

\section{My Birth, Name, Mother and Father}

So where do I begin the story of my life as of 10 July 2020? This often asked question usually has the answer: at the beginning of course, so here goes. I was born in the Glace Bay General hospital, Brookside Street, Glace Bay, on 20 December 1937. (This hospital building was torn down during the 1990s). In 1937 the world was in the middle of the Great Depression. Perhaps awful rather than great would have been a better adjective to describe the depression as it would more clearly describe what were tough economic times for the poor working class to which my parents belonged. I was named Thomas James (Jim or Jimmie) after my father Thomas James (Tommy or Tom) who was named after his father Thomas James (Tom) who emigrated from Coatbridge Scotland in 1908. He left his life as a coal miner in the pits near the Glasgow area of Scotland to seek his fortune in the coal mines of Cape Breton. A year or so later he arranged for his wife Elizabeth, nee Brand, and five children to join him in Glace Bay. One can only imagine how desperate life must have been in Scotland at the time, such that it would drive coal miners to immigrate to the far distant Colony of Canada to seek a better life for their families. My father, their sixth and last child ,Thomas James, was born in Glace Bay on 6 November 1910.

/section{My Dear Mother}

My Mother, Selina Kathleen Dawe, known as Kay to all her family and many friends, and of course always “ Ma” to her 9 kids. Ma was born on 14 July 1918 in the Queen Charlotte hospital in Kensington, London, England . Her parents , father Walter Dawe a Canadian soldier from Conception Bay Newfoundland and Kathleen, née Hind , from Kensington. This was when the first World War was starting to wind down and the great Spanish Flu pandemic was about to begin its two years of worldwide sickness leading to an estimated 30 to 50 million deaths. (As I write this we are in the first wave of the Covid-19 Pandemic and anxiously awaiting a vaccine or an effective treatment for this disease which is particularly deadly to seniors like my wife Deanna and me.)

/section{My First Home}

Kay and Tommy, and I lived in a two room apartment on the lower floor of a two story house at Caledonia Crossings. This is the area in Glace Bay where Tommy grew up and got his first job in the pit at No. 11 colliery. This location was convenient as most of his relatives lived within walking distance as was the pit for work. Ma's family, the Dawes lived at the other end of Glace Bay which would be about 3 miles away, a good one hour walk or a ride on the railed tram cars. A grocery store, owned by father and son Jack and Elliot Atkinson, was nearby which was convenient as Kay had no ice box or fridge so had to buy meat and fish on the day it was needed. She continued dealing with this grocer even after she moved to where her parents lived, as she liked the owners and felt some loyalty to them. Almost all other food merchants in town were owned by Jewish families and around 1950 my mother decided to start dealing with one, the Black and White store, owned by Louis Cohen, as Tommy's nephew was a delivery driver for that store.

I have only vague memories about living in this first home. However I recall Ma on occasion looking back upon the first couple of years of married life with fond memories. She left a crowded household where life was tough and being the oldest she had lots of siblings to help care for. Her father, Walter, often made life very unpleasant especially after having too much to drink. The 4 years of horror he experienced in the trenches of France and Belgium in WWI no doubt were a major contributing factor in the cause of his outrageous outbursts of very upsetting behavior. Today such persons are classified as suffering from PTSD and receive treatment and compensation. Not so for WWI veterans. I'll address this aspect later. My aunts told me of how much they enjoyed visiting and spending a night with Kay and Tommy in their small but happy home. They always had fond memories of Tommy, their young handsome brother inlaw who dressed up in a suit with white shirt and tie to go get drunk on Saturdays.

/section{Tommy's Army Career}

However, our home started to get crowded when my brother Bob was born 5 March 1939, followed by brother Ray on 27 February 1941. This meant that 5 of us were living in a small two room apartment which would be considered very, very cramped by today's standards. Also, at this time Canada was involved in WWII against Hilter so in July 1941 Tommy felt the call and decided to do his patriotic duty by joining the Army. Tommy had never traveled far from Glace Bay, so when his buddies told him of the great adventure and companionship to be had in the Army it was probably irresistible. Tommy was thirty and had been working in the mines since age 16. He followed the example of most of the male adults he knew and learned his favorite pastime: hard drinking. The war time army life offered him the opportunity to do this limited only by the availability of the money to do it with. So off Tommy went and Kay stayed behind with three kids, two still in diapers.

Tommy had difficulty adjusting to military life. He took some training in Nova Scotia at Debert, Yarmouth, Aldershot and at Camp Borden in Ontario. While there he was away without official leave (AWOL) and spent 14 days in detention, and was previously confined to barracks for a similar period. He seemed to have trouble getting back to his base on time after short leave periods and by October 1943 it was decided that Tommy would be better employed in the war effort as a first class miner rather than as a soldier. He was granted what was called “miners leave” which was extended to the end of the war.

/section{We Move To A New Home}

A short time after Tommy's departure to the army my Grandfather Dawe and Ma decided that it would be best if she were closer to her family while Tommy was away. It must have also been of economic necessity because she surely didn't want to be back too close to her father and all the conflict it might entail. My grandfather converted his backyard garage into a small two room bungalow for my mother and three kids plus my father who seemed to manage weekends and other leave periods especially when his allowance ran out and he was thirsty. This small bungalow had no running water or toilet. A bucket on the kitchen table was filled from her parents kitchen tap and another pail behind the bedroom door served as our toilet facility. It was emptied in her parents' bathroom toilet daily which did have an indoor toilet but no tub or shower. My grandparents' yard and house became very busy places: Walter and Kitty with 7 of their own kids still at home and now in the backyard , daughter Kay with her three kids. No wonder Walter lost his cool from time to time!

/section{Life with Six in A Small Two Room Bungalow}

Tommy, now back to civilian life, was able to resume life as a miner with DOSCO ( Dominion Steel and Coal company) a firm with headquarters in England. He got work at No. 1B Colliery only a 20 minute walk from our home in Walter's back yard. At this time Walter was still working at No. 20 Colliery which was also nearby. Tommy celebrated his 33rd birthday on 6 November of 1943 and the next day Ma delivered him a fourth son, Nelson, as a belated birthday gift. In those days miners had no vacations or other benefits. Benefits such as vacations, sick days, paternal leave days, etc, were still decades away. While Ma recuperated for about 10 days in the General Hospital , Tommy continued working in the mine and one of Ma's sisters took care of my two brothers and me. When Ma came home with Nelson he would be the only one with a bed of his own, a small crib or bassinet, while Bob, Ray and me shared a bed and Ma and Da another bed. There probably was not a spare square foot of unused space in that small bedroom we shared. But it was cozy and we survived.

/section{Sept. 1943: School at New Aberdeen Public School}

I was enrolled with about 35 other kids in Verda Robertson's primary class. Primary was then what would later be known as kindergarten. It was almost a full school day: 9am to 12 noon, walk home for a one and a half hour lunch break, return for the afternoon session from 1:30pm to 2:30pm. My aunts who still attended the same school took me to and from school for a few days but before long I was walking to and from school on my own as their school hours went until 3:30pm.

Verda Robertson, my first teacher, was an unmarried lady who I believe spent her whole career teaching Primary class. When I started in her class, she must have been close to 50 years old . She was a much loved teacher and helped a lot of kids get a good start in school. In those days most of the public school teachers were unmarried females who had one year of normal school training after grade 12 graduation to qualify as teachers and were paid less than \$100 a month. Teaching and nursing in those days were called the caring professions and society seemed to consider that what they did was from their sense of dedication and duty thus it was not necessary for them to be paid a high salary. It was not until the early 1960s that this attitude changed and these professions began to catch up.

/section{Long Friendship With Ray Fiolek Begins}

In the Primary class we shared desks, my desk mate was Ray Fiolek. My friendship with him lasted all through school and beyond. His father , like mine, joined the army and it resulted in his early death in a battle with Germans in Italy. He was killed on 30 December 1943. In order to try to make ends meet, his mother Gladys, set up a small store in the front of their house. Ray invited me home one day when we were in grade 1 and I couldn't believe my good fortune to be invited to a home which had a built-in candy store with chocolate bars, pop, and best of all an ice cream cooler. His mother gave us double header ice cream cones which we ate as we lay on the floor looking at the comics in the newspaper. In those days my allowance was 5 cents which would buy one of: a single scoop of ice cream or a chocolate bar or a bottle of pop. Treats of ice cream and other goodies other than on Saturdays, allowance days, were a rarity. Ray had a younger brother Dale who went on to become a dentist, while Ray joined the RCMP. I'll have more to say about this later. Meanwhile economic necessity made it necessary for Gladys Fiolek to move less than one quarter mile to live with her mother, Mrs MacDonald, in a more spacious company house. Her mothers' kids were all long since grown up and her husband, Ray's grandfather, had died so she could now share her house with Gladys. Ray and Dale and mother stayed until they finished high school and Gladys stayed in the home until her death. This home was also one of my “homes away from home” when I got to my early teen years. 

/section{1944: Grade One with Miss Keigan, 1945 Grade Two with Miss MacLeod}

I have mostly fond memories of my school days. I enjoyed learning almost everything and that continues to this day. After Verda Robertson in Primary grade, I moved to grade 1 with Miss Keigan, another much loved teacher. In September 1945 I entered Florence MacLeod's grade 2 class. I remember starting adding and subtraction and each afternoon she would read a chapter of a story to us which I really enjoyed. I also remember bringing twenty five cents to class to contribute to the purchase of a \$25 Canada saving bond to help the war effort . On one occasion Miss MacLeod selected me to go to the post office nearby to buy the stamps for the bond. Of course this was the last year for such bond purchases as WWII ended that year.

/section{Grade 3 :1946, Miss Violet Dick}

Grade 3 was special for a few reasons. I started to notice pretty girls, can't remember why I did, just came naturally I suppose. The first to catch my eye was Shirley Hopewell, a shy pretty blond girl whose attention I often tried to get, without much success I might add. The other special event in grade 3 was that we started to have 3 sets of written exams : usually in December, March then finals in June. But if you had excellent marks in the first two sets of exams you would be accredited which meant you would be exempt from June finals and be promoted to the next grade. I was accredited every year from grade 3 to grade 8 inclusive, but not grade 9. In grade 3, I was not only accredited but I came first in my class, the only time I won such an honor in school. I received a \$1.00 prize donated by the Orange Lodge. A neighbor, Ruth Beaton, one of Ma's friends, asked Ma if she was excited about grading day when June came along . Ma said she didn't realize that it was grading day. She was always so occupied looking after her four sons and husband and was also pregnant with her first daughter Elaine. Later that day she talked to Ma and her words were something like: “So you didn't even know it was grading day and yet your son came first in grade 3 “. Her son, Donnie, was my friend and classmate , he passed but of course he did not come first. His mother Ruth had only two kids so had the luxury of being able to pay attention to such things as “grading day” when the kids brought home certificates if they were promoted to the next grade. In those days kids were sometimes required to repeat a grade so this day could be a tearful one for kids who didn't get promoted to the next grade. We didn't use the term promoted in those days, we simply called it grading as in you either graded or you failed.

/section{Sept. 1947:Grade Four With Miss Margie Beaton}

I only recall two significant things from grade 4. The teacher , Miss Beaton , was the prettiest of all my teachers. I believe I started to daydream more about her than Shirley Hopewell! And I guess I must have been misbehaving because one month when I brought home my report card with all the usual marks in the 80s, I had an “F” , a fair grade for conduct. Ma glossed over the excellent academic marks and in strong terms told me that I must do better in conduct next month. She could understand a person having difficulty with some school subjects but in her opinion everyone should be able to get a “G” , good, or better for conduct . She was right, and I did!

/section{Sept. 1948: Grade 5 with Miss Babe Shears}

I can't recall anything of significance in this grade except that I am not sure if the teacher was Miss MacLennan or Miss Babe Shears, not to be confused with my grade 6 teacher , her older sister Jean. I called one of my old classmates still living in Glace Bay, Norman Carmichael, in Nov. 2020 and asked him about the teacher we had. He wasn't sure either. I think it was Babe Shears unless I get a reason to think otherwise.

/section{Sept.1949: Grade 6 with Miss Jean Shears}

Miss Shears was another excellent teacher. Nearing age 12 I became more aware of my appearance. I remember pressing my pants almost every night and was becoming more aware of girls. This was the year that the MacGregor family moved to New Aberdeen and four of their daughters started to attend our school: Ann in my grade 6 class, Ethel 2 years behind and Margie another year behind Ethel. A fourth daughter was in grade 9 so not in my world .Ann was a serious student and high achiever who most of the guys liked but realized, like me,that she was beyond their reach. She would end up marrying another classmate ,Norman Carmichael, another high achiever. Ann's father was the Underground Manager in the mine where my father worked. Norman's dad was the Paymaster for the mines in Glace Bay. So I and most of the rest of the class lived in a world that was somewhat different from theirs. 

/section{Sept. 1950: Grade 7 with Miss MacLean}

This was a combined class: one row of grade 8s and the rest of us in Grade 7. The main thing I remember was starting manual training, one half day a week for all grades 7,8 and 9 in Glace Bay. Yes, girls were included, they went to Central School also for what was called domestic science classes.It was always in the morning from 9 to 12 and I loved it! So once a week I took the bus downtown to Central School to attend. It was divided into semesters with time spent in woodworking, sheet metal and drafting each year. I remember some of my shop teachers very well. Mr MacNutt, my first ever male teacher, never smiled, he taught woodworking . I believe he had several kids and his wife had died so his life was likely difficult. He was always sucking on a black Smiths' Brothers cough drop which I came to realize later was likely a substitute for tobacco. I still remember one of his first lessons: the difference between crosscut and rip saws. My most memorable manual training project was in his class in grade 9 when I made a mahogany coffee table for my Ma. I still remember him teaching us to rub the varnished finish with pumice stone to make the finish almost glass like. To me at the time it was almost like magic! Ma had that table in her frontroom for years afterwards

/section{Sept. 1951: Grade 8 with Laura Donaldson}

Laura Donaldson lived in our neighborhood and was an excellent teacher. I had her for grade 8, grade 9 and for some subjects in high school. She lived about 100 meters from our back yard but her family kept to themselves. On rainy mornings she would sometimes give us a ride to or from school, not often, just sometimes.She never married and eventually moved to Morrison Glace Bay High School and then to the Sydney Academy, likely a promotion for her. She unfortunately died early in her retirement, due to a brain tumor I believe.

/section{Sept. 1952: Grade 9 with Laura Donaldson}

There were 2 grade 9 classes: one taught by Miss Donaldson the other by our much feared principal,Miss Ferguson! I was lucky that I never had the latter as a teacher but I experienced her much dreaded red strap on one occasion. She wore glasses and had a very stern look almost always! I learned just recently that Anne MacGregor's older sister used to be upset almost every school morning at the thought of going to her class.

In grade 9 I remember two items. One was having been assigned to write a short essay about William Shakespeare. I looked him up in our Books of Knowledge at home and prepared my essay which was handed in to Miss Donaldson. A few days later as she handed them back she commented on mine as being the best and had me write it on the black board for the class to copy. The second item, not so good. During December exams I rushed my algebra exam in order to be able to leave class early and ended up with my first failing exam at 44\%! This was a shock to me and I resolved to do better in the second term and did so with a mark of 88\%. But to my great disappointment the failure in the first term prevented me from being accredited so I had to write final exams in June for the first time.

My social life started to improve in relation to girls.In late grade 8 and then grade 9 the teachers held evening dances for us with recorded music. This was a great idea and thanks to the lovely girls at our school who taught most of the guys how to dance: foxtrot, square dance and just a bit of jive, rock and roll that is. Thank goodness girls mature a bit faster than boys so most of them had learned to dance at home or at house parties with their girl friends. This was great preparation for high school and for my life ahead as in 1959 I met Deanna and asked her for a dance and 62 years later we're still together and still dancing, albeit at a slower tempo.

/section{Sept. 1953: Grade 10 at Morrison High School}

I always liked school and really enjoyed my high school years. For many people, grade 9 was the end of school. Many persons , especially those who had failed a grade, were now 16 and could legally leave school as many did to take up employment in retail, at the fishplant and of course for many others the coal mines. My father,understandably, was not one to plan and set objectives but he did encourage us on more than one occasion not to follow him into the coal pits! This unstated objective was met: none of his 5 sons entered the coal mine except years later as visitors to the Miners' Museum in Glace Bay. At the time, my likely vocations were being a member of the RCMP or, later, a jet pilot in the RCAF. Higher education was never mentioned in our home: my father had grade 4, and disliked school, and Ma got to grade 10 which was pretty good given that her own father could not read! At the time, a high school certificate was still something which could open the door to employment across Canada and only a small percentage of grade 12 graduates would carry on to university.

High School in grade 10, class C3,was a big change in many ways. First of all I had to rush each morning to catch the 8:30 bus to school for the 9:00 am start. The need to prepare a lunch ,which I took in my new red lunch pail, added to the time crunch each morning. This lunch would be eaten in the boys lunch room , sometimes with a bottle of pop as a beverage. We used to have a 90 minute lunch break, so for a short while I used to walk to my aunt Winnie's home to have a cup of tea with my lunch. Then for a while I went with my good friend Donnie MacIsaac to his sister Norma's home where she made lunch for him and served me tea to go with my lunch from home. When winter arrived it made sense to stay at school and use the lunch room which was what I did for the remainder of my high school years.

Another welcomed change was a variety of teachers, including male teachers, usually a different teacher for each subject. I had about 7 subjects in grade 10. I found that I had to work harder to keep up but I still didn't give it the effort I should have. I tried to drop Latin in the second semester but the principal wouldn't allow it as I was still passing in all subjects. My favorite subjects were physics and algebra. I remember looking in a physics book and seeing a diagram of a single piston engine: that was a eureka moment for me as I now understood the basic principle of the internal combustion engine! I always did my written assignments but did not study as I should have. It was not until I ran into difficulty with my electronics training in the RCAF that I finally realized that I had to study or I would fail. I'll amplify this later. Usually I could get by with what I learned in class and this allowed me to pursue other interests like pool, cards and hanging out with my close friends.

We started to play pool after school. We usually played partners with the losers paying for the pool games. Our foursome was Ray Fiolek, my longest childhood friend, Donnie McIsaac, my second best friend and Jim Baxter. We all loved playing pool and were very competitive. A game would last about 20 minutes and cost 20 cents, at the rate of a cent a minute.

Being a 15 year old, a full blown teenager, I really appreciated that we could now go to a dance almost every Friday night. One week the student council would organize a dance in the school auditorium with recorded music. On alternate Friday nights, Knox United Church organized Teen Town dances with live music! My friends and I , especially Don McIsaac, never missed a dance. The majority of students were not paired up with dates, only a small minority went steady, that is, had a steady girl/boy friend. The girls would sit around the perimeter of the dance floor, the guys would stand around then when the music started the fellows would walk across the floor to ask one of their favorite girls to dance. Most girls would not refuse but some of the “downtown” girls would only dance if one of their Mr. Rights asked them. They didn't appreciate how much courage it took for a teenage boy to walk across the dance floor only to be turned down. Fortunately, most of the guys learned quickly which girls to avoid in order to be spared the embarrassment of rejection.

/section{Sept. 1954: Grade 11 At Morrison High}

Now 16, in possession of a driver's license and experienced in high school life I entered grade 11, class B5. Acquiring my driver's permit will be explained later with my account of working for Louis Cohen at the Black and White store. This was a very important year as to successfully complete grade 11 required successful completion of provincial examinations leading to Junior Matriculation certificate. This was often the minimum requirement for entering some colleges, office jobs, applying for aircrew in the RCAF and so on. My favorite subjects were geometry ,chemistry and history including Greek and Roman history. This came in handy a few short years later, in 1959, when I toured Italy , including Rome , with Deanna, my wife of 7 months. The other significant event was attending the prom in June 1955.

The prom was a very special event: a rite of passage towards adulthood. My friends and I knew various girls who we danced with at school dances or house parties so we had to decide who to invite to this very special event. So Ray, Donnie and I decided we would go to the prom. An essential was transportation and I was the only one amongst my friends with a driver's license. In the summer of 1954, one Sunday I borrowed uncle Herb's 1936 Dodge to take Ma and the kids for a drive in the country. During the drive the brakes started to fail but I managed to get the car back home without any trouble! So in May of 1955 I managed to get up the nerve to ask my favorite aunt Nookie if she would ask her husband Herb Nash to let me borrow his newer car, a 1946 Ford, for the prom night. Herb was not particularly enthused at this request but Nookie was the boss so the prom night transportation was arranged. 
Now who to invite? It came down to two choices: Ethel McGregor or Catherine (Sis) Eveleigh. I made the wrong choice and invited Sis! She was in grade 9, as was Ethel, but she had competed in a Glace Bay wide public School Queen contest sponsored by the Kinsmen's or other service club. Ethyl was a nice girl who I had more dances with but my teenage brain went with the gal who at the time had the “ School Queen title”. Big Mistake! I should have invited Ethyl.

/section{Prom Night: June 1955}

When the special Prom day arrived I went down to see Uncle Herb to pick up his car about 3 or 4pm. His body English spoke volumes about letting this 17 year old nephew borrow his pride and joy 1946 Ford! But he handed over the keyes and off I went. After dressing with the new blue suit that Ma bought for me, I picked up Don and Ray and we went to Thoms' florist in downtown Glace Bay to pick up the corsages for the girls. But then the Ford didn't want to start, likely due to “flooding” the engine, a common problem in cars at that time. So just like a scene from a hollywood teen movie, there we were dressed up in suits, white shirts and bow ties, me driving and my two buddies pushing the car down Commercial Street , the main downtown street in Glace Bay, trying to get the temperamental Ford started: it worked, thank heavens. We then made the 10 mile ride to New Waterford to pick up Ray's date. I've forgotten who Donnie's date was but we picked her up then we went to pick up Sis who sat beside me in the front, of course, the other two couples shared the back seat. The dance was held in the auditorium of Central School and Emilo Pace and his orchestra provided the music. This was one of , if not the top band on Cape Breton Island. After the dance we went to the Silver Rail night club out on the Sydney to Glace Bay highway and had a meal sometime after midnight. Then after a brief attempt at necking out at the nearby Airport, we drove the girls home and by 3am I managed to hand over the keys to Herb who was still up and waiting. I think he planned to go fishing that morning.

/section{Sept. 1955: Grade 12 At Morrison High}

I successfully passed my provincial exams so now had my Junior Matriculation Certificate. My two best friends, Ray and Donnie, did not pass so had to repeat grade 11.

In September a very unfortunate incident occurred: my uncle Robbie Nauss, married to Ma's sister Margaret, took his own life with a hunting rifle. He was a carpenter who was building a small apartment building in Dartmouth and if my memory serves me correctly, a crack occurred in the foundation wall causing him great distress fearing that the sale of the building might not materialize. He left Margaret and 4 young children in a difficult position financially and emotionally. My grandfather took charge in arranging for a group to go up to the funeral which was a drive of about 260 miles, a long journey in those days. The group included Nookie , Herb and their 2 year old daughter Brenda, my two grandparents and me as a driver to assist Herb. In those days there were no multilane highways so the 260 miles was a long tiresome motor ride and Walter figured Herb could use some help as he had just finished a shift in the coal mine. We left in the afternoon, rain started, making night driving conditions worse.This was fine with me as I still got a kick out of driving, however it did interfere with my grade 12 studies.

Getting back to school, I found some courses more difficult in keeping up with the work. In late October, the idea of becoming a Jet Pilot in the RCAF became even more attractive to me. My friend Donnie, now repeating grade 11, was not overly keen on school so went with me to the RCAF recruiting office in Sydney to take the tests for the Air Force. I passed the test to qualify for the more comprehensive aircrew selection process in London , Ontario. Donnie did not pass the air force test, instead was offered the chance of joining the army but decided to pass it up and stay in school.

This decision to quit school with only 8 months left before graduation was a decision I would come to regret for years afterwards. However, my feelings now, as I write this, are that it all worked out for the best as I have lived a long and happy life with Deanna ,our family and friends. 

/section{Nov. 1955: Signing up for the RCAF}

In late mid November, after completing the application form, I took a train to Halifax to have my medical exams and meet with the recruiting officer. I stayed with my aunt Margaret in her Dartmouth upstairs apartment where she and her four children were still adjusting to life without late father Roby. Besides getting my medical exams I had an interview with the recruiting officer. He questioned me about some of the information on my application form. I had not been completely honest with some of the information. Specifically, for hobbies, I wrote that I was interested in radio controlled model airplanes. I found a model airplane while in Dartmouth attending my uncle Roby's funeral just two months earlier, in September of 1955. The plane's owner likely lost control of it and it came down to earth none the worse for the crash landing. There was no owner in sight so I picked the plane up and took it home to Glace Bay. I thought it would be helpful to record my interest in model airplanes, which I had, but could never afford. I thought it would look good on my Air Force application to have such a hobby. My second white lie was failing to list my part time job history.. This was an outrageous lie as I had a variety of part time jobs ever since I was 8 years old! I didn't think that any of my jobs would add anything to my resume! How naive I was. So the officer asked a very obvious question: how could you, the oldest child in such a large family, afford such a hobby when you didn't even have a part time job? What was I to do? Admit to the lies about my hobby and the omissions about all my part time jobs? I just sat there stirring and feeling embarrassed .This was to be my first lesson in the risk of telling lies , even when they are what one considers to be of the harmless and victimless little white lie variety. He also asked if I had any dress clothes. To which I replied yes as I had a new blue suit which I wore to the prom just four months earlier and also had a nice sports jacket that I purchased out of my grocery store earnings. He suggested I wear such clothes when I go to the Officers Selection recruitment centre in London Ontario. He added that I should have arrived in his office dressed as a serious job applicant. He was right of course: the Bay Bi learns another lesson!

Upon returning home I started to gather things I needed such as a suitcase. I went down to my grandparents where Nookie and Herb lived and they let me borrow a flight bag. It turned out to be a permanent loan as it traveled with me to Ontario, then eventually to Germany, England, etc. until it wore out.

In November, my YPU friends at Warden United Church, held a going away party for me. Young Peoples Union was the church group for teenagers. I received a Gillette safety razor and \$20 cash as gifts which were much appreciated. When I left Glace Bay I had about \$26 in my wallet.

Around 5 December 1955, on a Friday or Saturday evening, my parents arranged for a car ride to the Sydney train station. Up to this point, no one from school contacted me to try to have me reconsider my decision. Ma was a bit saddened but thought I knew best...which of course I didn't. My father said nothing about it until this evening at the train station. As hugs were now finished, and all present almost in tears, I turned to step up into the train when Da said: “ Stay home Bi, you don't have to go”. I did have to as my friends had held the going away party for me, so I stepped up onto the train and began my adventure into adult life. I didn't realize that my departure from Cape Breton at this time was to be permanent except of course for vacation visits which continue to this day. With the arrival of sister Marilyn in August 1955 our small home became even more crowded with a total of 7 kids plus parents for a total of 9 crowded around the kitchen table and accommodated in 4 small bedrooms. Being the oldest I believe I was more sensitive to the situation in our household and realized that another child was the last thing our home needed. So perhaps it was time to move on to make room for the new arrival. This negative feeling regarding more kids came back again in 1958 when a letter from home informed me that Ma gave birth to yet another child, sister Judy. To say I was unhappy to receive this news would be an understatement.
My Air Force life will be described in detail in another part of this biography.

/section{My Part Time Jobs from age 8 to 17}

When you are a child in a large family and your father digs coal for a living there is always a shortage of money. Nevertheless, we got weekly allowances beginning at five cents a week with increases as we got older. The five cents on Saturday would buy you a one scoop ice cream cone, or a chocolate bar or a bottle of pop: one of these, not all three. So there was an incentive to earn your own money especially when you got interested in the very popular Saturday afternoon movies at one of our two theaters. I had a variety of jobs during my childhood. A description of them follows.

/subsection{Collecting Empties}:

In the war years nothing was wasted including glass beverage bottles which would be reused until they were chipped or otherwise unfit for use. Many kids would always be on the lookout for empties, usually pop or beer bottles. I remember one summer day when my grandfather and uncles were drinking beer in the dining room of my grandparents house. In those days the beer usually came in quart bottles and the empties were worth 2 cents. I sat in the doorway and watched the drinking and as each empty was placed on the floor, I grabbed it and took it outside with the others. Even 3 empties could be exchanged in the corner store for one of my favorite treats. Unfortunately my grandma Dawe, who we called Nana, began to collect the bottles. How does a 6 or 7 year old compete with competition from a Grandma? My own father drank a lot every weekend but it was usually at the Legion or a beer parlor with his friends. Thus, no empties were to be collected from his drinking. The rum bottles he occasionally took home to drink were not returnable.

/subsection{Delivering Magazines and The Star weekly}

Around age 9, I began delivering a monthly magazine called Liberty, an American magazine but with a Canadian edition. I just checked Google and noted that an 8 July 1944 copy of Liberty was available for \$222.00! The newsstand price at the time was 5 cents. The Liberty route didn't last long before I got the chance to deliver The Star Weekly every Monday.

The Saturday Star Weekly edition of the Toronto Star took almost 2 days to get to Glace Bay so I delivered it usually on Monday after school. It was ten cents a copy and I got to keep two cents per paper. I believe the price increased to 15 cents and I got a fifty percent raise to 3 cents a paper! I recall my route having about 30 or 40 customers.What I really wanted was a paper route for the Cape Breton Post, which was delivered 6 days a week, afterschool, and would provide a nice weekly income of about two dollars per week. But families tended to pass the route along from one sibling to another, so it was almost impossible to get a route. Contrast that with today's affluent society where kids in most towns wouldn't be found dead delivering papers. Instead retired seniors, or usually immigrants, do this job.

This paper route introduced me to the importance of collecting your money! Many customers didn't have 15 cents on the day I delivered so would say something like :” Sorry dear,I don't have any change right now can you come back later”. In one case, the man was a coal company watchman who was paid monthly and that was how long I had to wait for my money, collecting 60 cents for 4 weekly deliveries . Fortunately, I only had to send the money to the Toronto Star once a month so that helped my cash flow concerns. There was one neighbor who had no children, had a car and the woman of the house was always dressed up in nice clothes but she always had trouble paying me .She still owes me for 5 weeks in arrears, 75 cents that I'll always have on my books. I learned that many households were broke shortly after payday. In our home it was not uncommon for Ma not to have one dollar in the house on Monday. She used to borrow small amounts from the cash I kept in a glass in the china cabinet, then would have to come up with two or three dollars at the end of the month when I had to send the money order to the Toronto Star. Each month she would vow not to borrow this way again...but she usually did.

subsection{Christmas and Valentine Cards}

The Star Weekly often had full color advertisements to entice kids to sell special occasion cards, especially those for Christmas and Valentine Days. The compensation was not usually cash, rather some item that was really attractive to kids . One time, the offer to sell Christmas cards earned an adjustable ring bearing either the Toronto Maple Leafs or Montreal Canadiens crest. I sold about 12 boxes of cards to neighborhood women who couldn't say no to Kay's cute little boy Jimmie with his big blue eyes! I had to wait for about 6 weeks or more for the ring to arrive, it seemed like an eternity! Finally, it was there when I came home at lunch one February day, so I adjusted it, proudly put it on my finger and showed it off to my brothers at home and later to envious classmates. It was made of soft metal, a lead alloy, and the adjustable part under my finger soon broke off repeatedly until the ring was no longer wearable. This was one of my early disappointments and my first lesson in learning that things aren't always as good as advertised.

Another time I sold Valentine cards, the reward gift was unbelievable: a 5 power telescope! Again on a sunny day in May when I came home for lunch the scope had arrived complete with a black simulated leather case. I promptly threaded the case onto my pants belt and I proudly set off for school with the scope in my hands or in the case. I still remember the thrill of having it magnify images making them appear closer or making them magically farther away when I reversed the scope. When outside playing I would store the scope safely on top of the kitchen cupboard. Then one sad day, I went to fetch it and it was gone! This was a big disappointment for me, especially since none of my three brothers had any idea of the scope's whereabouts. I have my suspicion of which brother took it but to this day that has not been confirmed.

/subsection{Collecting Coal}

About age 10 in 1947, during Easter break , my friend Gerald Gilbert and I wanted to go to the matinee. But we were broke. He lived about 400 yards from the railway tracks where the train would pass a couple of times a day to take loaded boxes of coal from 1B colliery up to the siding to await further rail transport to Sydney. With each passing train some coal would fall from the 60 ton boxes to the tracks below. So Gerald and I took an empty burlap potato bag, a homemade two wheel cart, then walked up to the tracks and began to pick coal. Usually the pieces were about the size of a golf ball more or less and it would take about 1 hour to fill our sack. The potato sack when full would hold about 75 pounds of coal which we could sell for about 25 or 30 cents. As we walked along the tracks we got closer to the colliery which was patrolled by company watchmen most of whom were kind and considerate of miners' kids trying to earn some money collecting coal, or junk. But one watchman, Peter Smith, was a miserable son of a bitch. On one occasion he walked up to us, took the bag of coal off the cart, slashed it open with his knife and scattered the coal down the banks of the elevated rail bed and told us to get off the property. There was no matinee that day!

Peter Smith was widely known and hated especially amongst us kids . And to make matters worse he stole much more from the coal company that was paying his salary than we kids ever did. He and his son had a pony and a nice two wheel dump cart which held about a half ton of coal. Donnie MacIsaac and I watched one early summer evening when Peter and his son took the Pony and wagon to where the full coal cars were waiting and proceeded to offload a cartload of coal. No picking coal piece by piece for him and his son! This account would not be complete without including about when the bastard Peter got his just desserts. One evening while he patrolled at number 20 colliery he approached a young man, a local tough, who was in the process of breaking into the warehouse where boots, shovels, etc.,were stored. The man struck Peter with the bar he was using and almost killed him. He ended up in the hospital and the would-be thief ended up in the Sydney county jail. There was little sympathy for Peter expressed when this story was reported in the local paper.

On those days when we successfully picked a bag of coal, we wheeled it down Connaught Avenue, to find a customer. We would contact those persons most likely to buy coal from us. Persons like Mrs Harris , a widow, and Mike Perehiniak who ran the convenience store were likely customers since they weren't coal company employees so had to pay substantially more for coal than miners. I seem to remember that when Da would pay \$3.00 for a half ton of coal the non miner purchasers would pay about \$5.00. So a bag of clean picked coal from us for 25 or 30 cents was a bargain. I remember on one occasion asking Mrs Harris if she wanted to buy coal for 30 cents a bag. She said :”Yes dear, but I can't pay you until the end of the month”. This was an offer I had to refuse as we needed the money now, not half an eternity away at the end of the month. We pushed our cart another 100 yards to Mike's store to sell it for 25 cents. Mike always had cash in his till but he usually bargained for a lower price of 25 cents which we gladly accepted as the matinee beckoned!

/subsection{Collecting Coal Big Time!}

A couple of years later at age 14, I moved up scale in collecting coal. Some technical background is required at this point. Miners loaded coal into boxes holding 2 tons , about 4000 pounds of coal. At the surface these were dumped into large coal cars that held 60 tons or more. These rail cars were emptied in Sydney to await transport via coal boats. The key point is that the box cars were seldom completely empty when they came back to the siding near the 1B mine. So often, on Saturday mornings when the weather was suitable, we would check the empties to see if they had enough coal to make it worthwhile. If the empty had coal we would climb inside, load the coal into a burlap sack, as much as we could carry, then climb up the sloping inside of the car and down the ladder built on the side of the boxcar, then walk about 100 yards to dump it. We continued this until we had what we estimated would be a full load, half ton, about 1000 pounds of coal. There would usually be two or three teams of kids doing this, one of whom included Peter MacIntyre who lived nearby and whose family owned a horse and wagon. His father Tommy, was a police sergeant and his uncle was the chief of police for Glace Bay, a guy us kids referred to as Mickey the chief. This name will come up once again in my journey.

Usually by early afternoon, having our load of coal,we would shovel it onto the wagon and go down Connaught Avenue to sell it to Mike, Mrs Harris or another customer for \$3.00. We paid Peter 50 cents for cartage leaving us \$2.50, \$1.25 each. As big time coal collectors we were now beyond Saturday matinee and could afford the regular evening movies. The next item on this busy Saturday's agenda was to go home, get a towel and soap and walk to the washhouse at 1B colliery for a shower often lasting one hour. What the hell , the coal company had what we considered an unlimited supply of hot water. We then went home, dressed, had supper then headed downtown to the movie and the treats we could afford to buy with our hard earned \$1.25. Incidentally, the coal Peter collected, usually would be taken home for the personal use of his family, headed by his father Tommy , the police sergeant. So there couldn't have been too much illegal about what we were doing, could there? Also, Tommy's brother Mickey, was chief of police and he used to buy some of his coal from bootleggers, people mining coal illegally. DOSCO had the monopoly on mining and selling of coal in most of Cape Breton. Any others mining coal were doing so illegally and called bootleggers. So at that time in Glace Bay we had one version of the law for the common folk and another for those who make their living from it: policemen, lawyers, judges. Is it any different today?

\subsection{Collecting Junk}

At the age of 11, I started to collect junk. Junk in this case was scrap metal mostly iron and steel being the more readily available metal which was worth about 25 cents per hundred pounds. The second category was lead, brass and copper but these were more limited in availability so we got between 5 and 15 cents per pound for those. 
My first junk collection buddy was Gerald Bateman who was about 2 years older and a head above me physically , a nice guy but not too bright. At that time I seemed to be attracted to big guys but I was smaller and looking back now I realize that maybe I was the brains of the outfit....or at least the mouth! Gerald and I collected steel pieces washed up on the shores of our nearby Atlantic beaches which were just about in our backyards. To be more precise, it was about one quarter mile from our yard to our favorite Wallace Road beach. We would also take old bed frames from neighbors, old car parts, etc. We stored the scrap alongside a coal shed in Gerald's yard. When the day came to sell , we phoned the junk dealer, Joe Lipkus, who came with a truck, picked up the pile of scrap along with Gerald and me and proceeded downtown to the junk yard where we offloaded the scrap onto a scale for weighing. I believe we had about 300 pounds, so got about 75 cents for the lot. We split the money and walked home rather than pay bus fare out of our earnings.

By 1951, age 13, I was in grade 8. Sometime during the year, I met Donnie (Duck) McIsaac who was to become my best friend for the remainder of my school days. He was a year older than me and taller by a head so we became like Mutt and Jeff (two comic strip characters one being much taller than the other). We met in the school yard and discovered we both shared an interest in collecting junk, or more correctly, we both had a need for more money and this was the only way we knew how to get it. Donnie was even harder up financially than I was. His father had died a few years earlier, just before his eleventh child was born. This left his mother, Effie, with 11 kids and no income except what widow's allowance paid, not much in those days. Also, around this time, I learned that war has an upside, at least for scrap metal collectors it did. The price of scrap iron and steel rose from 25 cents per hundred pounds to \$1.00 per hundred pounds after the Korean War erupted in June 1950.

For the next two years or so, Donnie and I collected most of our scrap iron and steel from the shore areas that surrounded 1-B colliery. This is why my home address was 11, 1B Road, because the road in front of our house ran down to the 1B colliery. My father had the distinction of living the closest to the pit which was an advantage from a transportation point of view as he could walk to work in less than 5 minutes. Da worked in 1B which would later be known as number 26 colliery where he finished out his mining career spanning 44 years.

The coal company would dump all its underground waste and damaged equipment into the ocean. The stormy atlantic would break it up and throw it back upon the beaches where we swam in July and August. In those months we occasionally built a raft and reclaimed sheets of steel that lay off shore in 2 or 3 feet of water.Other pieces lay upon the beach making easy pickings. This was hard work especially to get the pieces to the top of the cliff . Often there weren't basic foot paths meaning we sometimes took one step upwards and slid back one or two. We stored our junk at Donnie's home as it had a new basement that we could keep the precious junk in. Bear in mind, it was now worth \$1.00 per hundred or 1 cent a pound. Now with the higher prices, we would often collect as little as 200 or 300 pounds before calling the new junk man, Schieke Goldman. This would give us a dollar, more or less, each, enough for pocket money to buy pop and a hotdog, play some pool, and admission to the matinee, etc.

By the time we got to grade 9 we discovered the pool room, or the “rat hole” as some people called it. It was run by Greek brothers, Steve and John Markadonis. Their family owned the entire building with offices on the third floor, the very popular Markadonis shoe repair shop on the main floor and the pool room in the basement. So often, after selling our junk, we went to the pool hall. Steve made the best hot dogs in town, so as we arrived at his counter, the usual request was for pop, a hot dog and a request to rack 'em up Steve, the pool balls on the table that is. After a game or two of pool, each lasting approximately 20 minutes, it was time to head home. The pool cost 1 cent a minute and later when we entered high school we usually played losers pay, but at this time in grade 9 I think we shared the cost. At this point , assuming I got \$1.50 for the junk, I would have about \$1.00 left: hot dog and pop cost 20 cents, the pool another 20 cents or so. Donnie could never resist trying his luck on the pinball machine in the pool room. It cost 5 cents a game , was kind of fun to play and if you were really lucky you might win extra games. Donnie would sometimes win 10 , 20 or to 40 games which could be cashed in at 5 cents a game. But he would never cash in. He tried to win even more games and the result was always the same: we walked up the stairs to the street and he was flat broke, not even a dime left for bus fare home. Rather than loan him a dime for bus fare we would hitch hike or walk home. I learned early the risk of loaning money to a friend: don't, you may end up losing both!

/section{The Black and White Store and Louie Cohen}

In June ,1953 I was 15 and just about finished grade 9 so Ma thought I might be able to get a part time job for the summer. A few years earlier, Ma stopped buying groceries at Atkinson's and switched to the Black and White Store because Da's nephew, Johnny McInnis, was the delivery man for that store. Johnny had since moved on but Ma continued to deal with Louis Cohen who owned the store located in the Sterling, an area of town close to the downtown. Ma called Louie and asked if he could use summer help. The answer was yes, and as soon as school was finished in June, I began my first full time job.

I worked Monday to Saturday, a total of 54 hours, more or less, per week for which I was paid \$10 weekly. It was not a lot but I was happy to have the job and the income. None of my friends had summer employment, so I was fortunate to be employed. Louis , the owner, was always in the store except for his lunch hour which he took at home. Two women, Annie and Mary were full time clerks who took customer orders over the phone, made up the orders and prepared the written receipts. They also ran the cash register and waited on drop in customers. They were paid \$18 weekly. The driver, Dissy Murphy, delivered customer orders using a half ton dodge pickup truck. I honestly can't remember his proper name, Dissy is what everyone called him. He was paid \$22 per week. So in comparison to these experienced adults, my \$10 weekly seemed reasonable, especially when it was topped up by the stuff I ate on the job: slices of delicious cooked ham, chocolate bars, bananas, pop,lots of cookies, etc.

A 15 year old boy has a big appetite and can eat anytime. One of my favorites was the delicious cooked ham which I thinly sliced for customers, and when Louis wasn't looking , or he was at home for lunch, I helped myself to a slice or two. This ham sold for \$1.20 a pound, the most expensive item in the meat showcase. Ma never bought any of it, it was too expensive. This was the most tasty and tender ham I ever ate, before or since! At least that's how I remember it. While on the topic of prices, some other meat and fish items were priced as follows, in cents per pound: dried cod 29 , hamburger 39, sausages 39, stew 49, round steak 59, sirloin steak 69, t bone steak 79. I never saw or heard of filet mignon while working at the meat counter. Ma's usual selections were dry or fresh cod, herring, mackerel, hamburger, stew, sausages, pork chops, and occasionally round steak. Liver was also available, something one doesn't see in the grocery stores nowadays.

My duties were varied: I restocked shelves and repackaged potatoes. They came from the wholesaler in burlap sacks holding 75 pounds. I transferred them to 15 pound (one peck) brown paper bags, and wrapped them with string. I learned to cut, weigh and package meat,etc. which I enjoyed. Saturday was the busiest day, when most of our orders were received, made up and delivered to customers all over town. I assisted the driver by loading filled order boxes onto the truck and from the truck into the homes of customers where we would empty them onto kitchen tables and take the empty boxes away. While doing Saturday deliveries, Dissy would, from time to time, treat me to a delicious milkshake. He also made deliveries of smaller orders which customers would phone in during the week. Ma had to go to a neighbor to phone in her order as we didn't have a phone until after I left home.

On Saturday, the driver carried a money pouch around his waist and would take payments from customers . The miners would be paid in cash on Friday afternoon or evening. The customers did not always pay the full amount of their order and many maintained a running credit account. For example, if Da had a small pay, Ma might receive \$45 worth of groceries but pay only \$35 on her account. The other ten to be paid next week if Da had a bigger pay. Many miners worked for a daily pay of \$8.00 or \$40 weekly. Other miners, like Da , worked on contract and were paid by how much coal they loaded. This varied depending on whether Da and his buddy had a good established room in which to blast out and load coal or if they were starting a new location that required start-up efforts. This meant Da's pay could range from \$55 to \$85 per week. This made it difficult for Ma to plan purchases. At least \$40 a week went for food so this left little to buy shoes, clothes, pay utilities and of course, to buy coal from DOSCO to heat the house and cook the meals. We were one of the many poor, hard working miners' families that's for sure. But at the time, I can never remember thinking that we were poor as there were many families like ours and some were even much worse off.

/section{Louis Cohen: My First Boss}

Most of the merchants in Glace Bay were Jewish, as were a lot of other professionals such as doctors and dentists. Louis was brought up in Reserve, a village just outside Glace Bay where his family ran a small food store. His brother Joe who had a physical disability, a back problem, ran that store with his mother. Louis married an attractive woman, Manierva ,who I believe came from a well off Ottawa family. I say this as I think this may explain how he at one time could afford to set up three stores in Glace Bay and live in a big two storey house on Chapel Hill, the nicest place to live in Glace Bay. The Jews generally kept to themselves and I believe they felt superior to miners and their families. However, I got on well with Louis. I liked him and sensed he felt something special for me even if I was not so gentle a gentile. One time a customer visited the store and complained to Louie about a meat item that had been delivered. Louie told her : “I'm sorry Mrs Leonard, that order must have been filled by the young fellow”. Standing a few feet away I interjected that I had not filled that order. Nothing more was said until Annie Mae told me that I shouldn't haven't spoken up like that as it is better for me to accept the blame rather than Louie. This transfer of blame made no sense to me then, nor now, some 70 years later. Clearly , Louie did not hold it against me. Later some people, including my siblings, nicknamed me little Louie. When Louis was not behind the meat counter it would often be me wearing one of the white aprons that Louis provided, looking after the meat side of the business.

The era of self-serve grocery stores was just starting up in Glace Bay and by 1955 three families would dominate the grocery business: Shore family, Gordon family, Mendelssohn family. By the time I started working at the Black and White store, two of the three stores Louis had were closed and the future didn't look too bright for Louis' last store. At the end of the summer of 1953, Louis asked me to stay on rather than go back to school and begin grade 10. Many of the miners' kids did not continue on to high school. I had enough common sense to say no thanks to this offer. Ma was a bit miffed when she found this out and I can't say I blame her. I retained a part time job working for Louis,on Saturday, working 10 hours, from 8am to 8pm for \$2.00. This paid for my bus fare to school, 75 cents for 10 tokens, the rest was allowance for pool, a movie on Wednesday afternoon and a dance on Friday night.

From Sept. 1953 until June 1954 I worked each Saturday at the Black and White store as well as a few extra days at Christmas time. Towards the end of June, I began my second summer working at the store. The Driver at that time was Russel Metcalf who I had assisted the previous winter delivering orders on Saturday. In early July, Russel took the afternoon off to get married to Sadie. He now needed more income so took a job in the mine leaving Louis without a driver. As I was slicing some bacon, Louis sitting 15 feet away behind his desk asked: Jimmie , can you drive the truck? I answered in the affirmative, despite the fact that I had no license. I used to help the driver on Saturdays during the winter and he would teach me how to drive in low risk areas on our delivery route. Louis asked if I would like to be the new driver to which I again answered an enthusiastic yes! We promptly went to the licensing office for a test drive and I earned my chauffeur's license. My pay more than doubled to \$22 per week! Russel had been paid \$24 so Louis saved a couple of bucks a week by hiring me. 

/section{Sausage Making and Delinquent Accounts}

To say that I was happy being a delivery truck driver would be a gross understatement! Louis didn't know it, but I would have likely driven the truck for only \$10 a week. My family, as with most families at the time, did not own a car so this was my chance to learn driving skills. I would be one of the few persons in our extended family to have a driver's license. And more importantly, this would make me a somebody amongst my friends. Getting this driving job was a big deal for me. I still recall the time while delivering orders in our own neighborhood when I noticed my best friend, Donnie McIsaac, and other local kids hanging around near Jimmie Evan's store, a favorite spot to hangout. I pulled up to the stop sign at the top of Wallace Road, shifted into second gear, let the clutch pop out resulting in the rear wheels spinning on dry pavement creating that squealing sound which was music to my ears as I screamed past my friends and on down the road. That's what showing off was all about when I was 16 and knowing that at least, this once, I was the envy of my friends. I was the only one in our group who had a driver license and a paying job for the summer, driving a truck no less in Glace Bay in 1954.

I was a good candidate as a replacement driver as I knew the route: where all the customers lived including the ex customers who had delinquent accounts , that is money owing to Louis which in many cases is still owing to this very day. Besides delivering orders, I would pick up bulk groceries at the wholesaler if we ran out of an item. In Glace Bay, at that time, stores were closed on Wednesday afternoon . Also, high school students got the afternoon off. The idea being to give more time to study as the curriculum was more difficult. I, like many others, chose to go to a movie on Wednesday afternoon.

On some Wednesday afternoons I would drive Louis into Sydney to visit the meat wholesalers. On one occasion we visited the Rabbi. He cut the throats of several chickens in his backyard and let them run about. This is where I learned what running about like a chicken with its head cut off was all about! These chickens would be taken home for Louis' family use. They were suitable as they were now kosher, that is, killed by the Rabbi.

The next stop was the meat packer. I remember one time we got 4 or 5 wooden crates filled with fresh pork loins...except these were not so fresh. When we got back to the store Louis and I unpacked the pork, loin by loin, trimmed off the tainted green spots, then washed the whole loin with vinegar. We then took the crates of pork to the cold storage locker down by the wharf. As the months passed, from time to time a box would be withdrawn from cold storage and sliced into lean pork chops and sold to our customers. I made sure Ma did not order pork from Louis. I don't know what the rules were at that time, but I suspect it was prohibited to sell tainted pork, other meat or fish for human consumption.

On other Wednesday afternoons, Louis and I would make sausages. We kept a box in the cooler which had chunks of fat and meat trimmings. These would be put through the meat grinder along with canned juices, spices,etc., which Louis would pick off his shelves. I don't recall him following an exact recipe. A spout-like attachment was placed on the outlet of the grinder on which we would place the sheeps' intestine, better known by the euphemism “ sausage casings”. The grinder was turned on, the casings filled and we twisted them every 4 inches or so to create the individual sausages. I advised Ma not to buy sausages from Louis. She didn't. To this day I still have some trepidation when eating sausage-like foods...but I still do from time to time especially pizza with pepperoni.

My least favorite task was the one that took place on Saturday afternoon, that being collecting payments from delinquent accounts. These accounts ranged from as low as \$25 to as high as \$300 which at that time was a significant amount of money. Louis would give me a one page list with the names of 20 or 30 accounts and the amount owing. I knew I had to make these calls because sometimes he would follow up my call with his own personal telephone appeal for a payment. 

One call I remember went something like this. I drove the truck in the yard, walked to the door , all 130 pounds of me, standing about 5 feet six inches tall with my money pouch slung around my waist presenting a very unintimidating image to the miner whose door I now knocked upon. The man, Benny L, an Italian Canadian, answered the door dressed in pants and undershirt with a big gut and hairy chest in full view. I let him have my standard spiel:”Louie Cohen wants to know if you have anything for him today?”. He asked my name and I gave it to him. His reply:”McEwan, you tell Louie that when I get it , he'll get it, understand”? Yes sir , thanks. Then on to the next call. One name on the list was my father's nephew who owed about \$150 or so, I refused to call on him, something about being embarrassed to do so or something like that. I had very little success with collections from these accounts. 

/section{Louie Cohen Gets in Hot Water}

Some time during 1954, a bus traveling from Ontario to the maritimes crashed resulting in the death of some passengers including one male , a Mr McAdam, from Glace Bay. He left his wife and child in Glace Bay to seek work in the booming factories of Ontario. His wife , Eliza, lived in a small upstairs apartment less than 200 feet from the front door of the Black and White store. Thus , it was very convenient for her to buy groceries there.

Mrs McAdam lived payday to payday as did the majority of miners' families, so the death of her husband brought immediate financial hardship upon her. She needed assistance. Louie realized that she would likely receive a settlement from life insurance or some form of compensation from the bus company, so he extended credit to her, initially for food, but gradually extending to other needs such as furniture, clothes,etc. About a year or so later Mrs McAdam received the death benefit due her. Louie sought payment of the total debt owing which I believe was in the low thousands of dollars. It seems that she didn't wish to make full payment so wrote Louie a cheque for a lesser amount. It was later alleged that he had altered the cheque to make it for a larger amount. Mrs McAdam sought legal advice, Louis was charged and convicted and sent to Sydney county jail for 6 months. This was a victory for Mrs McAdam and a disaster for Louis and his family. He would suffer disgrace in the Jewish community and the adverse publicity would not likely bring more business to his grocery store.

Louis' brother Joe took over the day to day running of the store in Louie's absence. After serving his term, Louie soon after quit his grocery business and moved with his family to Ottawa. I heard at the time that he was to operate a bowling alley. This was likely connected with the business interests of his in-laws. Some 38 years later, in Ottawa where I was then employed, one morning at coffee break, one of my coworkers talked about when he used to be in a bowling league at a bowling alley run by Louie Cohen. I guess my first boss, Louie, really did make it to Ottawa as I had heard he intended to do . Hopefully, he was able to establish a new life for him and family in Ottawa. Despite what he was alleged to have done, increasing the cheque amount, which I agree was, without a doubt, the wrong thing to do, I knew that he did help that person in their time of need and it would have been good if she had paid him in full as was his due, then the whole mess would have been avoided.

/section{Warden United Church Influence}

The church was a big influence in my life, especially my early years growing up in Glace Bay, then again years later when we lived in Ontario. My father had very little religion, maybe none, in his upbringing but Ma had some. First was with the catholic nuns who got her enrolled in primary school. Then later, as a teenager, she made a connection with the United church where at one time she sang in the choir. I have a photo showing her with the Warden Church choir at what appears to be age 16.

The first recollection I have of church activity was when I started school and someone got me going to mission band. This was a church group for young kids in grades 1 and 2. It was held in Wesley church hall where all Warden church activities were held after the church building had burned down years earlier.

I took part in a Christmas concert when I was in grade 1. I was Little Jack Horner so I had to get up on stage and say the usual lines as I held a pie plate in my hand. The next significant thing I remember is when I finished grade 2, the summer of 1946, I was 8 years old. The Mission Band leader arranged for me to go to a 10 day summer camp at the United Church campsite at New Campbellton, on the Bras D'or lakes. My friend, Ray Fiolet, was also to attend. Attending this camp was one of the most memorable experiences of my youth. I loved this camp and went every year until I reached grade 8. 

The first two years I went to camp we went by bus to the wharf at Sydney harbor then boarded a two decker lake boat, called the Lakeview. Ray Fiolek did not show up to catch the bus but, just as the boat was about to sail, a car drove up to the boarding ramp and out jumped Ray in time to join us. We also had another group from a United Church in Sydney, Whitney Pier, which included a mix of black and white boys. We steamed out of Sydney harbor, turned left, sailed in the Atlantic for a short while, then turned left into the inlet leading to the Bras D'or lake. An hour or so later we docked at the New Campbellton pier from which we could look up and see the L shaped row of 8 cabins, each would accommodate 8 campers on straw mattresses. The site was beautifully framed on two sides by tree covered hills, too small to be called mountains but we did refer to them as such as they were the highest most of us had ever seen. We walked the half mile up to the site to begin our fun filled 10 day adventure.

Besides having the 8 cabins for sleeping , there was a cookhouse and dining hall staffed by volunteer ladies from the United Church. We had a daily routine starting with washing ourselves in the stream nearby each morning. That is face and hands only, and feet if they were stinky. In any case, it was not a naked whole body wash! That we got from swimming. The stream was marked so that the area for washing was downstream from the area we used to get water for drinking and cooking. After washing it was then to the dining hall for breakfast. Afterwards some chores had to be done: water to be carried from the stream to the cookhouse, milk to be fetched daily from the nearby farm, potatoes and other vegetables to be peeled, garbage properly disposed of, etc.

About 10am we gathered for the morning chapel, remember it was a United Church of Canada sponsored camp. The chapel was a small clearing beside the stream, surrounded by trees, and outlined by rocks which defined the chapel's perimeter. The main camp leader was usually a church deaconess who would lead the chapel service that lasted about half an hour. In this, my first year at camp, the Deaconess was Miss Bessy French who was Warden Church Deaconess. After chapel ,there would be some recreation or free period that would take us up to lunch time. After lunch, after dishes were washed,we would have our main afternoon activities such as swimming in the Bras D'or lake, hiking , organized games and races, etc.

My favorite hike was on the unpaved road that hugged the lake for about a mile to a small waterfall that cascaded down from the mountain. Here we could swim in the cold water that settled into a natural swimming pool at the base of the falls. This became an annual hike and that spot has become one of my most cherished memories. To this day, I visit this waterfall on each visit to Cape Breton. The front garden of our Stittsville home has a piece of pink granite which I took from this area and which Deanna had engraved “McEwan” as a father's day gift for me.ĺBain family if memory serves me. Here we could purchase one of our favorite treats, assuming we still had some spending money left. Upon returning to camp, as the sun set, the large campfire was lit and we gathered around it to learn favorite campfire songs: Quartermaster Store, Jacob's Ladder, Old MacDonald Had a Farm, Swing Low Sweet Chariot, etc. We learned to sing them as rounds which made it more fun. Towards the end, hot chocolate and soda biscuits were available as a snack. Then it was off to bed in our cabins which didn't have electric lights so we depended on flashlights. These were also needed when we went to the outdoor toilets.

One of the camp leaders was Frank Poole, who was an officer in the Canadian army. I believe he had just returned from overseas having served in World War II, then continued as a career soldier until the 1970s when he retired. I was probably one of the smallest and youngest of the campers and in one special event involving a climb to the top of the surrounding mountain it was supposed to be just for the older and bigger boys. But Frank seemed to have a soft spot for me and he agreed that I could join the group. I remember we got to the top , stopped at the lake up there and left a cache of items to mark our accomplishment. It wasn't Mount Everest, but to us it was significant.

One year after returning from camp I was staring out our front window and happened to say something like: Gee I wish I were still up in New Campbellton. Ma's feelings must have been hurt as she said something like : If that's how you feel then go back there and stay. This kind of startled me and at the time I did not understand why she was annoyed. I do now.

One year my brother Bob and I both attended the camp. One afternoon the activity was swimming races in the Bras D'or lake nearby. Well it seemed like we were the best swimmers at the camp. For each event it was Bob came first, Jim came second or vice versa. In some of the races we were the only swimmers. I guess our close proximity and frequent access to the beach enabled us to learn to swim well compared to most of the others. Remember , in those days there were no swimming pools in Glace Bay , nor any Red Cross sponsored swim lessons. Kids learned on their own and older ones helped the younger and so on. Ma gave us a lot of freedom during summer holidays. This was decades before the term “ helicopter parents” was coined. The Atlantic ocean, at our doorstep, became our summertime playground where many of our hours were spent without the impediment of adult supervision.

On rare occasions, Ma packed a lunch and came to the local beach with us to watch us swim. My father never came with us. This was not unusual in those days. Parents were occupied with their work , and never ending household tasks and didn't have time for recreation and play which were considered mostly for kids. My father's main recreation was drinking with his buddies on the weekend. Drinking was very deeply ingrained in the life of many, not all, but many coal miners. This was the case with my father, his father and many of our relatives. I learned early about the misery, unhappiness and waste of precious financial resources that alcohol caused in many families like ours.

/section{Cubs, Scouts and Young People's Union}

About age 8, I moved on from Mission Band to Wolf Cubs, or simply Cubs. It was for boys ages 8 to 11 and started by Lord Baden Powell in 1916 to accommodate boys too young to join his Boy Scouts. Meetings were held once a week immediately after school lasting until 4:30pm. The objectives were to teach moral values, learn basic life skills, and of course have some fun with sports like dodgeball, knot tying competition,etc. The motto: Do a good turn everyday. We wore a uniform consisting of a green cap with a short peak, a knitted green pullover shirt with 2 buttons, and a red kerchief around our necks with a circular leather slider to hold it in place. As we learned knot tying, semaphore, first aid, and other basic skills, we were awarded badges. These were sewn to the sleeves of our cub shirt and worn with pride. The wolf pack was divided into groups of sixes, with a Cub, called a Sixer, in charge of each. Two yellow stripes were sewn on the arm of the shirt to identify Sixers. This was a way of starting leadership training at an early age, typically by age 10.

One of the very enjoyable Cub activities was Saturday morning hockey at the Miners' Forum , the only indoor artificial ice surface in Glace Bay. I would get up about 6am, have breakfast, take my stick and hockey bag and walk to the rink close to 2 miles away, near downtown. Occasionally I would take the bus, but they didn't run too often on early Saturday mornings. The game would start at 7 or 8 am and was usually supervised by Mr. O.B. Smith, the supervisor of schools for Glace Bay. He was a physical giant of a man, maybe the tallest in our town of 25,000 and obviously had a big heart as well for the youth of Glace Bay. On occasion he sent his son Benjamin to supervise and referee the game. We would always play a Cub pack from another church and it was run as a free for all, that is, all players were on the ice at the same time rather than the usual 6 aside. There were very few offside calls and the only face offs were when a goal was scored. So all the cubs present had an hour of full out skating even if some of them didn't touch the puck once with their sticks. A bird's eye view of the rink would have been like watching a large swarm of bees in pursuit of a moving source of nectar, as a total of 30 to 40 young Cubs pursued the puck carrier. Only young boys of 8 or 9 who loved hockey, and were thrilled to play on the nice big smooth ice surface of the forum would appreciate how much fun it was.

At age 11, I advanced to Boy Scouts, a very popular international organization for boys founded by Sir Robert Lord Baden Powell in 1907. The scout aims and objectives were like those for Cubs. The motto was : Be Prepared. The uniform was a smart Stetson style hat, a green cotton buttoned up shirt and, like the cubs, a kerchief. I never owned a hat, couldn't afford one.

Boy Scouts was for boys from age 11 upward to age 16 or more with some persons remaining in it as adult leaders for years. My best boyhood friend, Ray Fiolek, took Scouts very seriously and was the only one in our Scout troop to achieve the honor of Queen Scout.(Started as King Scout but changed when Queen Elizabeth took the throne in 1952). This was a big honor as Ray had a free trip up to Halifax to get the badge and certificate awarded to Queen Scouts. Years later, when Ray was an RCMP officer in Elliot Lake Ontario, he was very active in the Scout movement.If my memory serves me correctly, he was instrumental in getting a meeting hall built which was named Fiolek Hall in his honor.

I was in grade 9, my friends and I were now senior scouts so our Scout Master, Gordon Poole decided we were ready to do an overnight campout on our own, no adult supervision. I believe it was a day in September 1952, on a Friday afternoon, after school when he drove 4 of us out to a wooded area just outside the Caledonia area of Glace Bay. He departed intending to return on Saturday afternoon to pick us up. We went about the business of making a camp fire to warm up our supper, likely wieners and beans, then constructed a lean-to as shelter for overnight. I can't recall having sleeping bags but I have to assume we had them or equivalent. Unbeknown to us Mr. Poole came back later that evening to check on us without making his presence known. What he saw and heard from the four Scouts was , to put it mildly, very upsetting to him.

Some background on Gordon Poole is required at this point. He was the second of two sons of an upright living couple: no drinking, swearing, attended church every Sunday, etc. He graduated from university and was a high school teacher at Morrison High School where the four Scouts would attend in another year. I guess you could say Gordon led a sheltered life unlike his older brother, Frank, who joined the army and served in WWII and was very much a man's man. Gordon was a soft spoken gentle soul who did make a significant contribution to my church social life and to the lives of quite a few others who were in the church related groups which he led. For this, I have always felt a sense of real gratitude towards him, despite this incident.

Back to the incident. Instead of approaching us and “dealing with” his concerns about our unsatisfactory behavior he reported his concerns to our church minister, Reverend Crowe. Reverend Crowe was not the best fit as a minister for our church as he was too concerned with souls and less about our earthly bodies. It came to light later that he didn't prefer Cubs and Scouts as they didn't have enough religious instruction associated with them. He preferred new groups, namely Tyros and Sigma C, which eventually he got started in Warden United Church. I believe my younger brothers were in these groups.

In any case, Rev. Crowe called a meeting in his church office with me and the other 3 offending Scouts and in a calm voice proceeded to lecture us about the bad behaviors reported by Gordon Poole who he said was distraught, shocked and greatly disappointed at what he saw and heard from a distance on that recent Friday night: Boy Scouts swearing and smoking! I plead guilty to the swearing along with at least 2 of the others, but I did not smoke, two of the others did. My biggest regret about this affair was that it likely provided ammunition for Rev. Crowe to use in his successful attempt to replace the Cubs and Scouts with more religion based Church groups, Tyros and Sigma C. Upon reflection, I realize that the homes that Rev. Crowe and Gordon Poole were brought up in were worlds apart from those that a lot of miners' sons were reared in.

/section{Young People's Union (YPU)}

Around the time we entered grade 10, first year of high school, Donnie McIsaac, Ray Fiolek and I joined YPU. This was a group of 25 to 30 male and female students from grade 10 up to 12 who met one evening a week in our church hall under the leadership of Gordon Poole and his wife. We shared discussions, played games and learned some square dances.These things provided an alternative to cigarettes, beer and other diversions that persons in this age group were inclined to seek out. It also helped us get better acquainted with the opposite sex. YPU did, without a doubt, make a healthy contribution to our teenage lives . This in no way implies that we were angels or goody goody two shoe types , some were, but more of us weren't! But YPU was a positive influence to perhaps counter the negative influences which were all around us. This was the group that held a farewell party for me when I joined the Air Force two years later.